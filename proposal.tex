% This is a template for your written document.
%
% To compile using latexmk on the command line, run the following: 
% latexmk -pdf main.tex

\documentclass[12pt]{article}
\usepackage{setspace}
\usepackage{graphicx} % used for includegraphics
\singlespace
\usepackage[left=1in,right=1in,top=1in,bottom=1in]{geometry}

\title{\textbf{Project Topic}}
\author{Sam Smith}

\begin{document}

\maketitle

The VST (Visual Studio Technology) and AU (Audio Units) plugin ecosystem contributes greatly to the music industry today.
With the rise of digital audio workstations (DAWs) in the early 2000s, technology has become an integral part of music production and composition \cite{rana2024influence}.
In particular, the electric guitar has long been associated with technological advancements in the industry.
Engineers first harnessed the power of vacuum tubes to create the earliest electric guitar amplifiers, and later transitioned to transistor-based designs.
Today, traditional amps and effect pedals have been overtaken by virtual amplifiers and software-based effects plugins. Once-dominant traditional hardware companies
are now competing with highly technical and lucrative software companies that specialize in music production software \cite{rana2024influence}. The purpose of this project
is to design and implement a high-demand audio plugin for modern music production software. The plugin in question is a pitch shifter, a widely used
effect that digitally alters an input signal in real time. Its core functionality is to increase or decrease the input signal’s pitch by algorithmically altering every
frequency component in a signal \cite{rai2019analysis}. Although pitch-shifting effects already exist in both pedal and plugin form,
their widespread use and high degree of customizability make this a worthwhile and technically meaningful project to pursue.

When you use an image, such as in Figure~\ref{fig:method}, refer to it in the text.

\begin{figure}[h]
\begin{center}
% \includegraphics[scale=0.7]{methodology.png}
\caption{Archie}
\label{fig:method}       % Givee a unique label
\end{center}
\end{figure}


\newpage
\section*{Appendix}
A concise list of features / user stories in the order in which they will be built. A few examples are below to demonstrate the expected scope and level of granularity; you will have more features than this.
\begin{itemize}
	\item Default picture display on web application.
	\item On a button-click, user can separate the image into foreground and background.
	\item User can select a picture from their desktop.
	\item Selected picture displays on the web application.
\end{itemize}


\bibliographystyle{acm}
\bibliography{bibliography.bib}

\end{document}
