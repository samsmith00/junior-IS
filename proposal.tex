% This is a template for your written document.
%
% To compile using latexmk on the command line, run the following: 
% latexmk -pdf main.tex

\documentclass[12pt]{article}
\usepackage{setspace}
\usepackage{graphicx} % used for includegraphics
\singlespace
\usepackage[left=1in,right=1in,top=1in,bottom=1in]{geometry}

\title{\textbf{A Software-Based Pitch-Shifting Plugin for Modern Music Production}}
\author{Sam Smith}

\begin{document}

\maketitle

The VST (Visual Studio Technology) and AU (Audio Units) plugin ecosystem greatly contributes to the music industry today.
With the rise of digital audio workstations (DAWs) in the early 2000s, technology has become an integral part of music production and composition \cite{rana2024influence}.
In particular, the electric guitar has long been associated with technological advancements in the industry.
Engineers first harnessed the power of vacuum tubes to create the earliest electric guitar amplifiers, and later transitioned to transistor-based designs.


Today, traditional amps and effect pedals have been overtaken by virtual amplifiers and software-based effects plugins. Once-dominant traditional hardware companies
are now competing with highly technical and lucrative software companies that specialize in music production software \cite{rana2024influence}. The purpose of this project
is to design and implement a high-demand audio plugin for modern music production software. 

The plugin in question is a pitch shifter, a widely used effect that digitally modifies an input signal in real time. 
Its core functionality is to increase or decrease the input signal’s pitch by algorithmically altering every
frequency component in a signal \cite{rai2019analysis}. More importantly, it allows for a guitar player to quickly change the 
tuning of their guitar. This solves the common problem of a guitar player wanting to play a particular song in a certin tuning 
but their guitar does not support said tuning. Thus, the pitch shifter solves this problem by digitally modifying the guitar signal 
to the desired tuning before it reaches the amplifier. This is especially useful for guitarists who want to detune their instrument (tune the guitar to a lower pitch). 
Before detuning a guitar, one must consider a number of physical factors, such as the strings’ gauge (thickness), 
the width of the nut, and the length and tension of the neck. If even one of these factors does not support a major detuning, 
the guitar cannot be safely or effectively retuned. A pitch shifter removes these physical limitations by solving the tuning problem 
through a purely software-based approach.


Although pitch-shifting effects already exist in both pedal and plugin form, their widespread use and high degree of customizability 
make this a worthwhile and technically meaningful project to pursue. The plugin will have a simple interface that abstracts much of the 
complicated signal-processing logic and mathematics that occur behind the scenes. The program’s graphical user interface (GUI) will feature a dial that 
represents the semitone shift applied to the signal. Positive values will indicate an increase in pitch (uptuning), while negative values will indicate a decrease in pitch (detuning).

Additionally, the GUI will include a real-time visualizer of the incoming audio signal. This visualizer will display the signal 
in the frequency domain, with frequency measured in hertz along the x-axis and magnitude along the y-axis. 
This will provide  visual feedback in real-time showing how the pitch-shifting algorithm alters the guitar signal.

The core pitch-shifting algorithm will depend on several important parameters. According to Bowen Tang and Kiyofumi Tanaka in 
\textit{An Efficient Real-Time Pitch Correction System via Field-Programmable Gate Array}, key factors include latency, floating-point arithmetic, 
and the choice between time-domain and frequency-domain analysis methods \cite{tang2024efficient}. 
Notable algorithmic candidates for this project include the phase vocoder, constant-Q time-frequency analysis, and auto-correlation-based approaches.


%When you use an image, such as in Figure~\ref{fig:method}, refer to it in the text.

\begin{figure}[h]
\centering
\includegraphics[width=\textwidth,height=0.6\textheight,keepaspectratio]{imgs/plugin.png}
\caption{Proposed plugin design. It consists of a dial used to adjust the pitch of the input signal
as well as a visualizer to view the modified signal in real time. The graph's units are Hertz on the y-axis and magnitude along the x-axis}
\label{fig:method}
\end{figure}



\newpage
\section*{Appendix}
A concise list of features / user stories in the order in which they will be built. A few examples are below to demonstrate the expected scope and level of granularity; you will have more features than this.
\begin{itemize}
	\item Default picture display on web application.
	\item On a button-click, user can separate the image into foreground and background.
	\item User can select a picture from their desktop.
	\item Selected picture displays on the web application.
\end{itemize}


\bibliographystyle{acm}
\bibliography{bibliography.bib}

\end{document}
